\documentclass[twocolumn]{article}
\usepackage[utf8]{inputenc}
\usepackage[top=1in]{geometry}
\usepackage{graphicx}
\usepackage{booktabs}
\usepackage{amsmath}
\input{sym}
\title{Homework 8}
\author{Max marks: 120}
\date{Due on Nov 10th, 2021, 9 AM, before class.}
\newtheorem{prob}{Problem}

\newcommand{\bw}{\bar{w}}
\newcommand{\bx}{\bar{x}}
\newcommand{\by}{\bar{y}}
\newcommand{\bz}{\bar{z}}
\newcommand{\bA}{\bar{A}}
\newcommand{\bB}{\bar{B}}
\newcommand{\bC}{\bar{C}}
\begin{document}

\maketitle
\begin{prob}
  Derive a minimal state table for an FSM that acts as a three-bit parity
generator. For every three bits that are observed on the input w during three
consecutive clock cycles, the FSM generates the parity bit p = 1 if and only if
the number of 1s in the three-bit sequence is odd (10 marks)~\cite[Prob 6.12]{brown2013fundamentals}.
\end{prob}

\begin{prob}
  Design a modulo-6 counter, which counts in the sequence 0, 1, 2, 3, 4, 5, 0,
1, . . . . The counter counts the clock pulses if its enable input, w, is equal
to 1. Use D flip-flops in your circuit (20 marks)~\cite[Prob 6.23]{brown2013fundamentals}.
\end{prob}


\begin{prob}
  Design a three-bit counterlike circuit controlled by the input w. If w = 1,
then the counter adds 2 to its contents, wrapping around if the count reaches 8
or 9. Thus if the present state is 8 or 9, then the next state becomes 0 or 1,
respectively. If w = 0, then the counter subtracts 1 from its contents, acting
as a normal down-counter. Use J-K flip-flops in your circuit (20
marks)~\cite[Prob 6.26]{brown2013fundamentals}.
\end{prob}

\subsubsection*{State Reduction}
\begin{prob}
  Reduce the following state table to a minimum number of states:\\
  \includegraphics[width=\linewidth]{fig-15.2.png}\\
  (10 marks).
\end{prob}

\begin{prob}
  Digital engineer B. I. Nary has just completed the design of a sequential circuit
  which has the following state table:\\
  \includegraphics[width=\linewidth]{fig-15.3a.png}\\
  His assistant, F. L. Ipflop, who has just completed this course, claims that
his design can be used to replace Mr. Nary’s circuit. Mr. Ipflop’s design has
the following state table:\\
\includegraphics[width=\linewidth]{fig-15.3b.png}\\
\begin{enumerate}
\item Is Mr. Ipflop correct? (Prove your answer.)(10 marks)
\item If Mr. Nary’s circuit is always started in state $S_0$ , is Mr. Ipflop correct? (Prove
  your answer by showing equivalent states, etc.) (10 marks)
  \end{enumerate} .
\end{prob}

\subsubsection*{State Assignment}
\begin{prob}
  \begin{enumerate}
  \item Reduce the following state table to a minimum number of states using implica-
    tion charts (10 marks).
  \item Use the guideline method to determine a suitable state assignment for the
    reduced table (10 marks).
  \item Realize the table using D flip-flops (10 marks).
  \item Realize the table using J-K flip-flops (10 marks).
  \end{enumerate}\\
  \includegraphics[width=\linewidth]{fig-15.25.png}
\end{prob}


\bibliography{main}
\bibliographystyle{plain}
\end{document}
