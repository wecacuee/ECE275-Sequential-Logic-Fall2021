\documentclass[twocolumn]{article}
\usepackage[utf8]{inputenc}
\usepackage[top=1in]{geometry}
\usepackage{graphicx}
\input{sym}
\title{Homework 1}
\author{Max marks: 80}
\date{Due on September 8, 2021, before class.}
\newtheorem{prob}{Problem}

\newcommand{\bx}{\bar{x}}
\newcommand{\by}{\bar{y}}
\newcommand{\bz}{\bar{z}}
\begin{document}

\maketitle

\begin{prob}
Use algebric manipulation to prove that $(x+y)\cdot(x+\bar{y}) = x$. \cite[Prob 2.2]{brown2013fundamentals} [10 marks].
\end{prob}

\begin{prob}
Determine whether or not the following expressions are valid, i.e., whether the left- and
right-hand sides represent the same function.
\cite[Prob 2.7]{brown2013fundamentals}[10 marks]
\begin{enumerate}
    \item $x_1 \bx_3 + x_2 x_3 + \bx_2 \bx_3 = (\bx_1 + \bx_2 + x_3)(x_1 + x_2 + \bx_3)(\bfx_1 + x_2 + \bx_3)$
    \item $(x_1 + x_3)(\bx_1 + \bx_2 + \bx_3)(\bx_1 + x_2) = (x_1 + x_2)(x_2 + x_3)(\bx_1 + \bx_3)$
\end{enumerate}
\end{prob}

\begin{figure}
\centering
\includegraphics[width=\linewidth]{fig-2.24a.png}
\caption{A three-input circuit}
\label{fig:fig-2.24a}
\end{figure}

\begin{figure}
    \centering
    \includegraphics[width=\linewidth]{fig-2.23.png}
    \caption{A three-variable function}
    \label{fig:fig-2.23}
\end{figure}

\begin{prob}
Draw a timing diagram for the circuit in Figure~\ref{fig:fig-2.24a}. Show the waveforms that can be observed on all wires in the circuit.\cite[Prob 2.8]{brown2013fundamentals}[10 marks]
\end{prob}

\begin{prob}
Use algebraic manipulation to find the minimum sum-of-products expression for the function $f = x_1x_3 + x_1\bx_2 + \bx_1 x_2 x_3 + \bx_1 \bx_2 \bx_3$. ~\cite[Prob 2.12]{brown2013fundamentals}[10 marks]
\end{prob}

\begin{prob}
Use algebraic manipulation to find the minimum sum-of-products expression for the function $f = x_1\bx_2\bx_3 + x_1x_2x_4 + x_1\bx_2 x_3\bx_4$.~\cite[Prob 2.13]{brown2013fundamentals}[10 marks]
\end{prob}

\begin{prob}
Represent the function in Figure~\ref{fig:fig-2.23} in the form of a Venn diagram and find its minimal
sum-of-products form.~\cite[Prob 2.17]{brown2013fundamentals}[10 marks]
\end{prob}


\begin{prob}
Design the simplest sum-of-products circuit that implements the function $f (x_1 , x_2 , x_3 ) = \sum m(3, 4, 6, 7)$.~\cite[Prob 2.21]{brown2013fundamentals}[10 marks]
\end{prob}

\begin{prob}
Design the simplest product-of-sums circuit that implements the function $f (x_1 , x_2 , x_3 ) = \prod M (0, 2, 5)$.~\cite[Prob 2.22]{brown2013fundamentals}[10 marks]
\end{prob}


\bibliography{main}
\bibliographystyle{plain}
\end{document}
