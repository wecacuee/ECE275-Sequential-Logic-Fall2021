\documentclass[twocolumn]{article}
\usepackage[utf8]{inputenc}
\usepackage[top=1in]{geometry}
\usepackage{graphicx}
\usepackage{booktabs}
\usepackage{amsmath}
\input{sym}
\title{Homework 6}
\author{Max marks: 50 marks}
\date{Due on Oct 25th, 2021, 9 AM, before class.}
\newtheorem{prob}{Problem}

\newcommand{\bw}{\bar{w}}
\newcommand{\bx}{\bar{x}}
\newcommand{\by}{\bar{y}}
\newcommand{\bz}{\bar{z}}
\newcommand{\bA}{\bar{A}}
\newcommand{\bB}{\bar{B}}
\newcommand{\bC}{\bar{C}}
\begin{document}

\maketitle
\begin{prob}
  Show how the function $f (w_1 , w_2 , w_3 ) = \sum m(0, 2, 3, 4, 5, 7)$ can be implemented using a
  3-to-8 binary decoder and an OR gate.~\cite[Prob 4.1]{brown2013fundamentals}
  (10 marks)
\end{prob}

\begin{prob}
  Consider the function $f = \bw_2 \bw_3 + w_1 w_2$.  Derive a circuit for
  $f$ that uses only one 2-to-1 multiplexer and no other gates (assuming inputs
  are available in both uncomplemented and complemented form).~\cite[Prob 4.4]{brown2013fundamentals}(10 marks)
\end{prob}

\begin{prob}
  Figure~\ref{fig:seven-seg-display} shows the notation for a BCD to 7-segment
display and Table~\ref{tab:seven-segment-tt} shows the corresponding truth
table. The inputs corresponding to the missing rows in the truth table should be
considered as don't care.
  \begin{enumerate}
   \item implement segment ``a'' using an 8:1 mux and no other logic gate, (10 marks)
   \item implement segment ``a'' using a 4:1 mux and one other gate, (10 marks)
   \item implement segment ``f'' with 4:1 mux and no other logic gate. Assume
     inputs are available in both uncomplemented and complemented form. (Hint:
     There are (${}^4C_2 = 6$) possible pairs of control inputs: ($w_3, w_2$), ($w_2, w_1$),
     ($w_1,w_0$), ($w_0, w_3$), ($w_0, w_2$), ($w_1, w_3$). There are 6 don't
     care conditions. With two control inputs of the multiplexer and one input,
     you can represent an expression with up to 4-SOP-terms of size
     three-literals or less. You might the arrive at the answer sooner, if you
     try to write the minimal SOP expression first and find the two inputs that
     occur most often in all the terms. Those two inputs are most likely to be the
     chosen pair of control inputs.) (10 marks)
  \end{enumerate}
\end{prob}

\begin{figure}[h!]
  \includegraphics[width=\linewidth,trim=0 10cm 0 0.2cm,clip]{fig-4.21.png}
  \\
  \caption{Seven segment display and BCD-to-7-segment display converter. When
    $a=1$ the corresponding segment of the display lights up. To display the
    number 8, you will turn on all the seven segments, while to display 1, you
    will turn on $b=1, c=1$ and turn off $=0$ the rest. The full truth-table for
  the seven-segment display is shown in Table~\ref{tab:seven-segment-tt}.}
  \label{fig:seven-seg-display}
\end{figure}

\begin{table}[h!]
  \footnotesize
\begin{tabular}{l|cccc||ccccccc}
  \toprule
  Row & $w_3$ & $w_2$ & $w_1$ & $w_0$ & a & b & c & d & e & f & g \\
  \midrule
  0  & 0    & 0   &   0 &   0 & 1 & 1 & 1 & 1 & 1 & 1 & 0 \\
  1  & 0    & 0   &   0 &   1 & 0 & 1 & 1 & 0 & 0 & 0 & 0 \\
  2  & 0    & 0   &   1 &   0 & 1 & 1 & 0 & 1 & 1 & 0 & 1 \\ 
  3  & 0    & 0   &   1 &   1 & 1 & 1 & 1 & 1 & 0 & 0 & 1 \\ 
  4  & 0    & 1   &   0 &   0 & 0 & 1 & 1 & 0 & 0 & 1 & 1 \\ 
  5  & 0    & 1   &   0 &   1 & 1 & 0 & 1 & 1 & 0 & 1 & 1 \\   
  6  & 0    & 1   &   1 &   0 & 1 & 0 & 1 & 1 & 1 & 1 & 1 \\ 
  7  & 0    & 1   &   1 &   1 & 1 & 1 & 1 & 0 & 0 & 0 & 0 \\ 
  8  & 1    & 0   &   0 &   0 & 1 & 1 & 1 & 1 & 1 & 1 & 1 \\
  9  & 1    & 0   &   0 &   1 & 1 & 1 & 1 & 1 & 0 & 1 & 1 \\
  \bottomrule
\end{tabular}
\caption{Truth table for BCD to seven-segment display as shown in
  Figure~\ref{fig:seven-seg-display}. The missing combinations of inputs should
  be considered as dont care.}
\label{tab:seven-segment-tt}
\end{table}

\bibliography{main}
\bibliographystyle{plain}
\end{document}
