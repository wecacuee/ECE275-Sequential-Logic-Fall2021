\documentclass[onecolumn]{article}
\usepackage[utf8]{inputenc}
\usepackage[top=1in]{geometry}
\usepackage{graphicx}
\usepackage{booktabs}
\usepackage{amsmath}
\input{sym}
\title{Homework 4}
\author{Max marks: 40}
\date{Due on Oct 1st, 2021, 9 AM, before class.}
\newtheorem{prob}{Problem}

\newcommand{\bx}{\bar{x}}
\newcommand{\by}{\bar{y}}
\newcommand{\bz}{\bar{z}}
\newcommand{\bA}{\bar{A}}
\newcommand{\bB}{\bar{B}}
\newcommand{\bC}{\bar{C}}
\begin{document}

\maketitle
\begin{prob}
  Hazard problem: Design a hazard free SOP for $f(A,B,C,D) = \sum m(0,1,4,5,6,7,9,11,14,15)$
\end{prob}

\begin{prob}
  Find the simplest realization of the function $f (x_1 , \dots, x_4 ) = \sum m(0, 3, 4, 7, 9, 10, 13, 14)$,
  assuming that the logic gates have a maximum fan-in of two.
\end{prob}
\begin{prob}
  Find the minimum-cost circuit for the function $f (x_1 , \dots, x_4 ) = \sum m(0, 4, 8, 13, 14, 15)$.
  Assume that the input variables are available in uncomplemented form only. (Hint: Use
  functional decomposition.)
\end{prob}
\begin{prob}
  Use functional
  decomposition to find the best implementation of the function $f (x_1 , \dots,
  x_5 ) = \sum m(1, 2, 7, 9, 10, 18, 19, 25, 31) + D(0, 15, 20, 26)$. How does your implementa-
  tion compare with the lowest-cost SOP implementation? Give the costs.
\end{prob}


%\bibliography{main}
%\bibliographystyle{plain}
\end{document}
