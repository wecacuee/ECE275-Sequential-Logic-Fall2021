\documentclass[twocolumn]{article}
\usepackage[utf8]{inputenc}
\usepackage[top=1in]{geometry}
\usepackage{graphicx}
\usepackage{booktabs}
\usepackage{amsmath}
\usepackage{tikz}
\usepackage{multirow}
\usetikzlibrary{matrix,shapes,arrows,fit,tikzmark}

\input{sym}
\title{Homework 4}
\author{Max marks: 40}
\date{Due on Oct 1st, 2021, 9 AM, before class.}
\newtheorem{prob}{Problem}

\newcommand{\bx}{\bar{x}}
\newcommand{\by}{\bar{y}}
\newcommand{\bz}{\bar{z}}
\newcommand{\bA}{\bar{A}}
\newcommand{\bB}{\bar{B}}
\newcommand{\bC}{\bar{C}}
\newcommand{\bD}{\bar{D}}
\newcommand\marktopleft[1]{%
  \tikz[overlay,remember picture] 
  \node (marker-#1-a) at (.3em,.3em) {};%
}
\newcommand\markbottomright[2]{%
  \tikz[overlay,remember picture] 
  \node (marker-#1-b) at (.1em,.3em) {};%
  \tikz[overlay,remember picture,inner sep=1pt]
  \node[draw={#2},rounded corners,fit=(marker-#1-a.north west) (marker-#1-b.south east)] {};%
}

\begin{document}

\maketitle
\begin{prob}
  Hazard problem: Design a hazard free SOP for $f(A,B,C,D) = \sum m(0,1,4,5,6,7,9,11,14,15)$
\end{prob}
\subsubsection*{Solution}
The K-map for $f$ is
\\
\begin{tabular}{cc|cccc}
  \toprule
  && \multicolumn{2}{c|}{$\bA$} & \multicolumn{2}{c}{$A$}
  \\
  && $\bB$ & \multicolumn{2}{|c|}{$B$} & $\bB$
  \\ \midrule
  \multirow{2}{*}{$\bC$} & $\bD$
                                &\marktopleft{p1c1}1&\marktopleft{p1r1}1& 0 & 0
  \\\cmidrule{2-2}
  & \multirow{2}{*}{$D$}
                                  &\marktopleft{p1o2} 1\markbottomright{p1o2}{orange}&1\markbottomright{p1c1}{cyan}&0&\marktopleft{p1br1} \marktopleft{p1o1}1\markbottomright{p1o1}{orange}
  \\\cmidrule{1-1}
  \multirow{2}{*}{$C$}   &
                                  &0&\marktopleft{p1g1}1&\marktopleft{p1b1}1&1\markbottomright{p1b1}{blue} \markbottomright{p1br1}{brown}
  \\\cmidrule{2-2}
  & $\bD$
                                  &0&1\markbottomright{p1r1}{red} & 1\markbottomright{p1g1}{green}& 0
  \\\bottomrule
\end{tabular}

\[
  f = \color{red} \bA B + \color{green} BC + \color{blue} ACD + \color{cyan}
  \bA \bC + \color{brown} A\bB D + \color{orange} \bB \bC D
\]

\begin{prob}
  Find the simplest realization of the function $f (x_1 , \dots, x_4 ) = \sum m(0, 3, 4, 7, 9, 10, 13, 14)$,
  assuming that the logic gates have a maximum fan-in of two.
\end{prob}

\begin{prob}
  Find the minimum-cost circuit for the function $f (x_1 , \dots, x_4 ) = \sum m(0, 4, 8, 13, 14, 15)$.
  Assume that the input variables are available in uncomplemented form only. (Hint: Use
  functional decomposition.)
\end{prob}
\begin{prob}
  Use functional
  decomposition to find the best implementation of the function $f (x_1 , \dots,
  x_5 ) = \sum m(1, 2, 7, 9, 10, 18, 19, 25, 31) + D(0, 15, 20, 26)$. How does your implementa-
  tion compare with the lowest-cost SOP implementation? Give the costs.
\end{prob}


%\bibliography{main}
%\bibliographystyle{plain}
\end{document}
