\documentclass[twocolumn]{article}
\usepackage[utf8]{inputenc}
\usepackage[top=1in]{geometry}
\usepackage{graphicx}
\usepackage{booktabs}
\usepackage{amsmath}
\input{sym}
\title{Homework 7}
\author{Max marks: }
\date{Due on Nov 1st, 2021, 9 AM, before class.}
\newtheorem{prob}{Problem}

\newcommand{\bw}{\bar{w}}
\newcommand{\bx}{\bar{x}}
\newcommand{\by}{\bar{y}}
\newcommand{\bz}{\bar{z}}
\newcommand{\bA}{\bar{A}}
\newcommand{\bB}{\bar{B}}
\newcommand{\bC}{\bar{C}}
\begin{document}

\maketitle
\begin{prob}
  Consider the timing diagram in Figure~\ref{fig:p51}. Assuming that the D and Clock inputs shown
  are applied to the circuit in Figure~\ref{fig:5.10}, draw waveforms for the
  $Q_a$ , $Q_b$ , and $Q_c$ signals. (10 marks)~\cite[Prob 5.1]{brown2013fundamentals}
  \label{prob:1}
\end{prob}
\begin{figure}[h]
  \includegraphics[width=\linewidth]{fig-p5.1.png}
  \caption{Inputs for Prob~\ref{prob:1}}
  \label{fig:p51}
\end{figure}
\begin{figure}[ht!]
  \includegraphics[width=0.7\linewidth]{fig-5.10.png}
  \caption{Circuit for Prob~\ref{prob:1}.
    Recall that $Clk$ with a ``$\triangleright$'' symbol
    indicates rising-edge (positive-edge) triggered flip-flop. $Clk$ without
    ``$\triangleright$'' symbol indicates a level-triggered latch.
    $Clk$ with ``$\circ\triangleright$'' symbol
    indicates a falling-edge (negative-edge) triggered flip-flop. 
  }
  \label{fig:5.10}
\end{figure}

\begin{prob}
  Figure~\ref{fig:5.4} shows a S-R latch built with NOR gates. Draw a similar
latch using NAND gates. Derive its characteristic table and show its timing
diagram. Assume that $\bar{S}$ and $\bar{R}$ are available.
\label{prob:2} (10 marks)~\cite[Prob 5.2]{brown2013fundamentals}.
\end{prob}
\begin{figure}[ht!]
  \includegraphics[width=\linewidth]{fig-5.4.png}
  \caption{Circuit for Prob~\ref{prob:2}}
  \label{fig:5.4}
\end{figure}

\begin{prob}
  An SR flip-flop is a flip-flop that has set and reset inputs like a gated SR
latch. Show how an SR flip-flop can be constructed using a D flip-flop and other
logic gates. (10 marks)~\cite[Prob 5.5]{brown2013fundamentals}
\end{prob}

\begin{prob}
  Show how a JK flip-flop can be constructed using a T flip-flop and other logic
  gates (10 marks)~\cite[Prob 5.7]{brown2013fundamentals}.
\end{prob}

\begin{prob}
  Design a three-bit up/down counter using T flip-flops. It should include a control input
  called Up/Down. If Up/Down = 0, then the circuit should behave as an up-counter. If
  Up/Down = 1, then the circuit should behave as a down-counter. You can use a
  2-1 MUX for the switching or you can uses basic logic gates (10
  marks)~\cite[Prob 5.15]{brown2013fundamentals}.
\end{prob}

\begin{prob}
  The circuit in Figure~\ref{fig:p5.3} looks like a counter. What is the
  counting sequence of this circuit (10 marks)~\cite[Prob 5.17]{brown2013fundamentals}?
  \label{prob:6}
\end{prob}
\begin{figure}[ht!]
  \includegraphics[width=\linewidth]{fig-p5.3.png}
  \caption{Circuit for Problem~\ref{prob:6}.}
    \label{fig:p5.3}
\end{figure}

\bibliography{main}
\bibliographystyle{plain}
\end{document}
