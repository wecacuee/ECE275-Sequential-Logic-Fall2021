\documentclass[twocolumn]{article}
\usepackage[utf8]{inputenc}
\usepackage[top=1in]{geometry}
\usepackage{graphicx}
\usepackage{booktabs}
\usepackage{amsmath}
\input{sym}
\title{Homework 5}
\author{Max marks: 85}
\date{Due on Oct 15th, 2021, 9 AM, before class.}
\newtheorem{prob}{Problem}

\newcommand{\bx}{\bar{x}}
\newcommand{\by}{\bar{y}}
\newcommand{\bz}{\bar{z}}
\newcommand{\bA}{\bar{A}}
\newcommand{\bB}{\bar{B}}
\newcommand{\bC}{\bar{C}}
\begin{document}

\maketitle
\begin{prob}
  Determine the decimal values of the following unsigned numbers ($()_b$
  indicates a base $b$ representation) (25 marks):
  \begin{enumerate}
  \item $(0111011110)_2$
  \item $(1011100111)_2$
  \item $(3751)_8$
  \item $(A25F)_{16}$
  \item $(F0F0)_{16}$
  \end{enumerate}
\end{prob}

\begin{prob}
  Determine the decimal values of the following 1’s complement binary numbers
  (15 marks):
  \begin{enumerate}
  \item 0111011110
  \item 1011100111
  \item 1111111110
  \end{enumerate}
\end{prob}

\begin{prob}
  Determine the decimal values of the following 2’s complement numbers (15 marks):
  \begin{enumerate}
  \item 0111011110
  \item 1011100111
  \item 1111111110
  \end{enumerate}
\end{prob}

\begin{prob}
  Convert the decimal numbers 73, 1906, -95, and -1630 into signed 12-bit numbers in the
  following representations (20 marks):
  \begin{enumerate}
    \item Sign and magnitude
    \item 1’s complement
    \item 2’s complement
  \end{enumerate}
\end{prob}

\begin{table}
  \centering
  \begin{tabular}{ccc|c|c}
    \toprule
    $c_i$ & $x_i$ & $y_i$ & $c_{i+1}$ & $s_i$ \\
    \midrule
    0 & 0 & 0 & 0 & 0 \\
    0 & 0 & 1 & 0 & 1 \\
    0 & 1 & 0 & 0 & 1 \\
    0 & 1 & 1 & 1 & 0 \\
    1 & 0 & 0 & 0 & 1 \\
    1 & 0 & 1 & 1 & 0 \\
    1 & 1 & 0 & 1 & 0 \\
    1 & 1 & 1 & 1 & 1 \\
    \bottomrule
    \end{tabular}
  \caption{Truth table for Full adder}
  \label{tab:full-adder}
\end{table}

\begin{figure}
\includegraphics[width=\linewidth]{fig-3.4.png}
\caption{A decomposed implementation of the full-adder circuit.}
\label{fig:decomposed-full-adder}
\end{figure}

\begin{prob}
  Show that the circuit in Figure~\ref{fig:decomposed-full-adder} implements the
  full-adder specified in Table~\ref{tab:full-adder} (10 marks).
\end{prob}


%\bibliography{main}
%\bibliographystyle{plain}
\end{document}
